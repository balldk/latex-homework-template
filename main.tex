% ============ Import Library ============
% Apply template
\documentclass[a4paper,oneside]{article}
% Using vietnamese
\usepackage[utf8]{vietnam}
\usepackage[vietnamese=nohyphenation]{hyphsubst}
\usepackage[vietnamese]{babel}
\usepackage[utf8]{inputenc}
% Some essential libs for math
\usepackage{amsmath, amsthm, amsfonts, amssymb}
\usepackage{mathtools}
\usepackage{cases}
\usepackage{commath}
\usepackage{mathrsfs}
\usepackage{enumerate}
% Essential libs for paper format
\usepackage{graphicx}
\usepackage{scrextend}
\usepackage[margin=1.0in]{geometry}
\usepackage[explicit]{titlesec}
% Others
\usepackage{fancyhdr}
\usepackage{lipsum}
\usepackage{tikz, tcolorbox}
% Font style
\usepackage{mathptmx}
\usepackage[T1]{fontenc}

% ============ Configs ============
% paper formats
\changefontsizes{12pt}
\setlength{\topmargin}{-0.8in}
\setlength{\textheight}{9.25in}
\renewcommand{\baselinestretch}{1.2}
\titlespacing{\section}{0pt}{12pt}{2pt}
\DeclareTextFontCommand{\textbfit}{\bfseries\itshape}
\setlength{\parindent}{0pt}
% Define commands
\renewcommand{\leq}{\leqslant}
\renewcommand{\geq}{\geqslant}
\newcommand{\nl}{\\[1.5mm]}
\newcommand{\nll}{\\[2.0mm]}
\newcommand{\nlll}{\\[3.0mm]}
\newcommand{\dlim}{\displaystyle\lim}
\newcommand{\N}{\mathbb N}
\newcommand{\Z}{\mathbb Z}
\newcommand{\Q}{\mathbb Q}
\newcommand{\R}{\mathbb R}
\newcommand{\C}{\mathbb C}
\DeclareMathOperator{\Ima}{Im}
% QED
\newcommand*{\QED}[1][$\square$]{%
    \leavevmode\unskip\penalty9999 \hbox{}\nobreak\hfill
    \quad\hbox{#1}%
}
\newcommand*{\QEDFill}{\null\nobreak\hfill\ensuremath{\blacksquare}}
% \newcommand{\startproof}{\begin{center}Giải\end{center}}
\newcommand{\startproof}{\vspace{-7pt}\noindent\textit{Chứng minh.}\enspace}

\makeatletter
\newenvironment{sqcases}{%
    \matrix@check\sqcases\env@sqcases
}{%
    \endarray\right.%
}
\def\env@sqcases{%
\let\@ifnextchar\new@ifnextchar
\left\lbrack
\def\arraystretch{1.2}%
\array{@{}l@{\quad}l@{}}%
}
\makeatother

\theoremstyle{definition}
\newtheorem{exercise}{Bài}

% ============ Meta data ============
\title{\textbf{Bài tập Giải tích 2 - Chương 1}}
\author{Nguyễn Đình Đăng Khoa (MSSV: 20110217)}
\date{\today}


% ============ Content ============
\begin{document}
\maketitle

\begin{exercise}
    Xét tính liên tục của hàm $f$ được định nghĩa bởi
    $$
        f(x) =
        \begin{cases}
            \dfrac{\sin^2 x}{x^2}, & x \neq 0 \\
            1,                     & x = 0
        \end{cases}
    $$
\end{exercise}
\startproof
You \textbfit{should} start to write some proves here, the following equations and symbols are just demo and not a real proves for anything.
\begin{align*}
    \displaystyle\lim_{x\to0} f(x)
    = & \lim_{x\to0} \dfrac{\sin^2 x}{x^2}
    = \lim_{x\to0} \left(\dfrac{\sin x}{x} \cdot \dfrac{\sin x}{x}\right)                             \\[1.5mm]
    = & \lim_{x\to0} \left(\dfrac{\sin x}{x}\right) \cdot \lim_{x\to0} \left(\dfrac{\sin x}{x}\right) \\
    = & \ 1 \cdot 1 = 1
\end{align*}
$\Rightarrow \displaystyle\lim_{x\to0} f(x) = f(0) = 1$.\\[1.5mm]
$\Rightarrow f$ liên tục tại x = 0.\\
Mặt khác, nếu $x \neq 0$ thì $f(x) = \dfrac{\sin^2 x}{x^2}$ xác định trên $\R$.\\[1.5mm]
Vậy $f$ liên tục trên $\R$.\QED

\begin{exercise}
    Cho đồng cấu nhóm cộng $f: \Q\longrightarrow\Z$. Chứng minh rằng với $n\in\N^*, f(1) = nf\left(\dfrac{1}{n}\right)$.
\end{exercise}
\startproof
Cho $n\in\N^*$, khi đó
\[
    f(1)=f\left(\dfrac{n}{n}\right)
    =f\left(\smash[b]{\underbrace{\dfrac{1}{n}+\dfrac{1}{n}+\cdots+\dfrac{1}{n}}_\text{n lần}}\right)
    =\smash[b]{\underbrace{f\left(\dfrac{1}{n}\right)+f\left(\dfrac{1}{n}\right)+\cdots+f\left(\dfrac{1}{n}\right)}_\text{n lần}}
    = nf\left(\dfrac{1}{n}\right)
\]\QEDFill

\begin{exercise}
    Chứng minh rằng
    $$
        \sum_{k=1}^{n} k = \frac{n(n+1)}{2}
    $$
\end{exercise}

\end{document}